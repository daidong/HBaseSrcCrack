\documentclass[
  article,
  a4paper,
  a4wide,
  %draft,
  smallheadings
]{book}

% Packages below
\usepackage{graphicx}
\usepackage{verbatim} % used to display code
\usepackage{hyperref}
\usepackage{fullpage}
\usepackage[ansinew]{inputenc} % german umlauts
\usepackage[usenames,dvipsnames]{color}
\usepackage{float}
\usepackage{subfig}
\usepackage{tikz}
\usetikzlibrary{calc,through,backgrounds}
\usepackage{fancyvrb}
\usepackage{acronym}
\usepackage{amsthm} % Uuhhh yet another package
\VerbatimFootnotes % Required, otherwise verbatim does not work in footnotes!
\usepackage{listings}

\definecolor{Brown}{cmyk}{0,0.81,1,0.60}
\definecolor{OliveGreen}{cmyk}{0.64,0,0.95,0.40}
\definecolor{CadetBlue}{cmyk}{0.62,0.57,0.23,0}
\definecolor{lightlightgray}{gray}{0.9}

\begin{document}
\lstset{
language=Java,                             % Code langugage
basicstyle=\ttfamily,                   % Code font, Examples: \footnotesize, 
keywordstyle=\color{OliveGreen},        % Keywords font ('*' = uppercase)
commentstyle=\color{gray},              % Comments font
numbers=left,                           % Line nums position
numberstyle=\tiny,                      % Line-numbers fonts
stepnumber=1,                           % Step between two line-numbers
numbersep=5pt,                          % How far are line-numbers from code
backgroundcolor=\color{lightlightgray}, % Choose background color
frame=none,                             % A frame around the code
tabsize=2,                              % Default tab size
captionpos=b,                           % Caption-position = bottom
breaklines=true,                        % Automatic line breaking?
breakatwhitespace=false,                % Automatic breaks only at whitespace?
showspaces=false,                       % Dont make spaces visible
showtabs=false,                         % Dont make tabls visible
columns=flexible,                       % Column format
morekeywords={__global__, __device__},  % CUDA specific keywords
}
\begin{lstlisting}
  ClassLoader loader = new URLClassLoader(cp.toArray(new URL[0]));
  Thread.currentThread().setContextClassLoader(loader);
  Class<?> mainClass = Class.forName(mainClassName, true, loader);
  Method main = mainClass.getMethod('main', new Class[] {
    Array.newInstance(String.class, 0).getClass()
  });
  String[] newArgs = Arrays.asList(args).
    subList(firstArg, args.length).toArray(new String[0]);
  main.invoke(null, new Object[] { newArgs });
\end{lstlisting}

\end{document}
